%%%%%%%%%%%%%%%%%%%%%%%%%%%%%%%%%%%%%%%%%%%%%%%%%%%%%%%%%%%%%%%%%%%%%%%%%%%%%%%%%%%%%%%%%%%%%%%%%%%%%%%%%%%%%%%%%%%%%%%%%%%%%%%%%%%%%%%%%%%%%%%%%%%%%%%%%%%
% This is just an example/guide for you to refer to when submitting manuscripts to Frontiers, it is not mandatory to use Frontiers .cls files nor frontiers.tex  %
% This will only generate the Manuscript, the final article will be typeset by Frontiers after acceptance.                                                 %
%                                                                                                                                                         %
% When submitting your files, remember to upload this *tex file, the pdf generated with it, the *bib file (if bibliography is not within the *tex) and all the figures.
%%%%%%%%%%%%%%%%%%%%%%%%%%%%%%%%%%%%%%%%%%%%%%%%%%%%%%%%%%%%%%%%%%%%%%%%%%%%%%%%%%%%%%%%%%%%%%%%%%%%%%%%%%%%%%%%%%%%%%%%%%%%%%%%%%%%%%%%%%%%%%%%%%%%%%%%%%%

%%% Version 3.1 Generated 2015/22/05 %%%
%%% You will need to have the following packages installed: datetime, fmtcount, etoolbox, fcprefix, which are normally inlcuded in WinEdt. %%%
%%% In http://www.ctan.org/ you can find the packages and how to install them, if necessary. %%%

\documentclass{frontiersSCNS} % for Science, Engineering and Humanities and Social Sciences articles
%\documentclass{frontiersHLTH} % for Health articles
%\documentclass{frontiersFPHY} % for Physics and Applied Mathematics and Statistics articles

%\setcitestyle{square}
\usepackage{url,hyperref,lineno,microtype}
\usepackage[onehalfspacing]{setspace}
\linenumbers


% Leave a blank line between paragraphs instead of using \\


\def\keyFont{\fontsize{8}{11}\helveticabold }
\def\firstAuthorLast{Stewart {et~al.}} %use et al only if is more than 1 author
\def\Authors{Terrence C. Stewart\,$^{1,*}$, Ashley Kleinhans\,$^{2}$ and Chris Eliasmith\,$^1$}
% Affiliations should be keyed to the author's name with superscript numbers and be listed as follows: Laboratory, Institute, Department, Organization, City, State abbreviation (USA, Canada, Australia), and Country (without detailed address information such as city zip codes or street names).
% If one of the authors has a change of address, list the new address below the correspondence details using a superscript symbol and use the same symbol to indicate the author in the author list.
\def\Address{$^{1}$Centre for Theoretical Neuroscience, University of Waterloo,
    Waterloo, ON, Canada
    \\
$^{2}$Mobile Intelligent Autonomous Systems group, 
      Council for Scientific and Industrial Research,
      Pretoria, South Africa}
% The Corresponding Author should be marked with an asterisk
% Provide the exact contact address (this time including street name and city zip code) and email of the corresponding author
\def\corrAuthor{Terrence C. Stewart}
\def\corrAddress{Centre for Theoretical Neuroscience, University of Waterloo,
    Waterloo, ON, Canada}
\def\corrEmail{tcstewar@uwaterloo.ca}




\begin{document}
\onecolumn
\firstpage{1}

\title[Embodied Neuromorphic Benchmarks]{Embodied Neuromorphic Benchmarks} 

\author[\firstAuthorLast ]{\Authors} %This field will be automatically populated
\address{} %This field will be automatically populated
\correspondance{} %This field will be automatically populated

\extraAuth{}% If there are more than 1 corresponding author, comment this line and uncomment the next one.
%\extraAuth{corresponding Author2 \\ Laboratory X2, Institute X2, Department X2, Organization X2, Street X2, City X2 , State XX2 (only USA, Canada and Australia), Zip Code2, X2 Country X2, email2@uni2.edu}

\maketitle

%%%%%%%%%%%%%%%%%%%%%%%%%%%%%%%%%%%%%%%%%%%%%%%%%%%%%%%%%%%%%%%%%%%%%%%%%%%%%%%%%%%%%%%%%%%%%%%%%%%%%%%%%%%%%%%%%%%%%%%%%%%%%%%%%%%%%%%%%%%%%%%%%%%%%%%%%%%%%%%%%%%%%%%%%%%%%%%%%%%%%%%%%%%%%%%%%%%%%%%%%%%%%%%%%%%%%%%%%%%%%%%%%%%%%%%
%%% The sections below are for reference only.
%%%
%%% For Original Research Articles, Clinical Trial Articles, and Technology Reports the section headings should be those appropriate for your field and the research itself. It is recommended to organize your manuscript in the
%%% following sections or their equivalents for your field:
%%% Abstract, Introduction, Material and Methods, Results, and Discussion.
%%% Please note that the Material and Methods section can be placed in any of the following ways: before Results, before Discussion or after Discussion.
%%%
%%%For information about Clinical Trial Registration, please go to http://www.frontiersin.org/about/AuthorGuidelines#ClinicalTrialRegistration
%%%
%%% For Clinical Case Studies the following sections are mandatory: Abstract, Introduction, Background, Discussion, and Concluding Remarks.
%%%
%%% For all other article types there are no mandatory sections.
%%%%%%%%%%%%%%%%%%%%%%%%%%%%%%%%%%%%%%%%%%%%%%%%%%%%%%%%%%%%%%%%%%%%%%%%%%%%%%%%%%%%%%%%%%%%%%%%%%%%%%%%%%%%%%%%%%%%%%%%%%%%%%%%%%%%%%%%%%%%%%%%%%%%%%%%%%%%%%%%%%%%%%%%%%%%%%%%%%%%%%%%%%%%%%%%%%%%%%%%%%%%%%%%%%%%%%%%%%%%%%%%%%%%%%%

\begin{abstract}

    Evaluating the effectiveness and performance of neuromorphic hardware is
    difficult.  It is even more difficult when the task of interest is an
    embodied task; that is, a task where the output from the neuromorphic
    hardware affects its future input.  However, these embodied situations
    are one of the primary potential uses of neuromorphic hardware.  To address
    this, we present a methodology for embodied benchmarking and a particular
    example of a benchmark for adaptive control of an arbitrary system.  These
    benchmarks are flexible, in that they allow researchers to explicitly
    modify the benchmark to identify particular task domains where particular
    hardware excels.  Furthermore, the benchmarks make use of a hybrid of
    real physical embodiment and a type of "minimal" simulation that has been
    shown to lead to robust real-world performance, allowing the same benchmark
    to be used by many researchers.

\tiny
 \keyFont{ \section{Keywords:} neuromorphic hardware, benchmarking, 
        minimal simulation, adaptive control, neural networks} %All article types: you may provide up to 8 keywords; at least 5 are mandatory.
\end{abstract}

\section{Introduction}

Neuromorphic hardware holds great promise for a wide variety of applications.
The combination of massively parallel computation and low power consumption
means that there is the potential to have complex algorithms running in
embedded processing situations, without being a significant drain on battery
life.  The question, of course, is to identify what sort of always-on or
interactive functionality is feasible with these devices.

To evaluate applications of this hardware, we need benchmark tasks.  These
tasks must allow us to compare across different instances of neuromorphic
hardware (and potentially across different algorithms implemented in said
hardware).  This allows us to quantitatively compare systems, letting 
researchers both measure the progress in the field and also directly
compare competing approaches.

In this paper, we focus on the development of \emph{embodied} benchmarks.
These are tasks where the output of the neuromorphic hardware \emph{influences
its own future input}.  This is in contrast to standard categorization or pattern
identification tasks, where the input is some fixed sequence, and the main
question is whether the hardware produces the correct output for each input
(or input pattern).

We believe embodied benchmarks should be of particular interest to
neuromorphic research, given that many of the applications of neuromorphic
hardware are in exactly this domain of embedded and interactive control
of robotic or other physical systems.  However, embodiment raises a number
of issues that complicate the development of such benchmarks.  Rather than
simply providing a data file of inputs and desired outputs, the benchmark
must either specify a full physical system for that embodiment, or it must 
provide software for a simulation of that system.  As we discuss below, either
approach is problematic, and addressing this difficulty is the primary goal of
this paper.


\section{Embodied Benchmarks}

An embodied benchmark task is one where the system we are studying has a 
two-way interaction with some sort of environment.  That is, the outputs from
the neuromorphic hardware are sent to the environment where they cause some
sort of effect, the results of which change the subsequent input.  For example,
the outputs might control the movement of a robot, which in turn affects the
sensory data received by the robot.

\subsection{Simulation versus Physical Instantiation}

To define such a benchmark, we need to be explicit about the embodiment.  If
it is a situation where a robot is to be controlled, we need to be explicit
about all the details of that robot.  What motors are there?  How are they
configured?  How strong are they?  What sensors are there?  Where are they
placed?  How accurate are they?  However, even if these questions are
answered, there is a fundamental problem in that \emph{other researchers
need access to that exact robot}.  If a benchmark is to be widely used, other
researchers developing their own hardware should be able to do their own 
testing.

Furthermore, using a physical robot also imposes significant practical
difficulties when performing extensive benchmark testing.  When testing, we
often want to run the same task over and over again, both for robustness and
to see the effects of varying parameters.  With a physical robot, this means
manually setting up the task, letting the test run, gathering the resulting
data, and then resetting the robot back to the initial state.  This means that
issues like battery life become problematic, and not just because there is a
limited amount of time available for testing.  As the battery level changes,
the performance of the robot itself can also change.  Futhermore, for any
rigorous testing of the benchmark, we will want to examine situations where
the system fails.  This means that some of the testing will involve parameter
settings that lead to poor behaviour, which could involve physical damage to
the robot itself.

However, \emph{not} using a real physical embodiment for testing is also
problematic.  First and foremost, without an actual real-world task, why
should anyone have any confidence that the performance on the benchmark is
reflective of the actual usefulness of the neuromorphic hardware?  It is
very widely known that \emph{simulations of robots} (or other physical systems) are
often much easier to control and better-behaved than the real thing.  The
field of robotics is filled with algorithms that work well ``in theory'', but
fail when run on actual hardware.  We do not want a benchmark that falls into
this trap of failing to generalize to real situations.

Furthermore, neuromorphic hardware has another constraint that severely limits
the possibilities of simulation.  The normal approach when a simulation is
too simple to reflect reality is to add details to the simulation itself.
Incredibly finely detailed simulations can be created, filling in all of the
details needed.  However, those resulting simulations \emph{cannot be run
in real-time}.  This is a fundamental problem, in that most neuromorphic
hardware is about real-time interactions, and there is no way to slow down
the hardware to match the simulated environment.  This means that even if
we spent the considerable amount of research effort needed to define a
simulated environment for an embodied benchmark, that simulation could not
be run fast enough to interact with most of the hardware that we would want
to test.

\subsection{Minimal Simulation}

The above considerations seem to indicate that even though using real-world
phyiscal hardware for benchmarking is problematic, it is still better than
using simplistic simulations which may not generalize to real tasks.  However,
we believe there is an alternate approach known as \emph{Minimal Simulation}
\cite{Jakobi97evolutionaryrobotics}.

First, we note that the problem faced here is remarkably similar to a problem
faced by the evolutionary robotics community.  In evolutionary robotics, the
goal is to use genetic algorithms to \emph{evolve} systems that can control
robots to perform various tasks.  These tasks can be as simple as navigation
and obstacle avoidance, but have also included complex eight-legged walking,
object identification, and visual tracking [refs???].

However, performing this evolution on real physical robots is problematic
for the same reasons that benchmarks on physical robots are problematic.  The
robots must be reset to the same state each time; they often involve
behaviour that can physically damage the robots; and they take a very long
time to run.  For this reason, attempts were made to evolve using simulated
robots.  However, the general finding was that algorithms that worked on the
simulated robots would not work when run on the real physical robots.  If
the simulations were improved, adding complex physical detail, then it was
possible to generalize to real behaviour; unfortunately, such complex
simulations would run slower than real-time [refs].

To address this problem, Nicholas Jacobi proposed the creation of
``minimal'' simulations.  These are simulations where there is variability
\emph{within the simulation itself}.  In other words, we make \emph{poor}
simulations, but ensure that the way in which it is poor is itself
variable.  We then make sure that the controllers work across that whole
range of variability.  ``Instead of trying to eliminate the differences between
simulation and reality, they are acknowledged, and mechanisms are put in place
to prevent evolving controllers from relying on them.'' [ref: jacobi thesis].

With this approach, it was possible to build minimal simulations that would
run faster than real-time and yet also be complex enough that if a system
could successfully control the simulation, it was also likely to successfully
control a real robot.  To achieve this, the simulations are made to be
unreliable in almost every respect.  For example, for a simulation of a simple
motor it would still be the case that if power is applied it would generally
try to spin, but the exact amount of torque, the amount of sensory noise,
the amount of time needed, the amount of static and dynamic friction, and so
on would all be randomly chosen.  A successful controller would have to deal
with this wide range of variability, and if it could handle that variability
then we would have reason to believe it could also handle the real system.

It is also worth noting that a minimal simulation \emph{only has to be a
good simulation for successful behaviour}.  That is, ``if we are evolving 
corridor following behaviour, the dynamics of the simulation might
differ wildly from those of reality if the controller hits a wall or
goes round in circles, but this does
not matter, since the controllers we are interested in transferring
across the reality gap will neither
hit walls nor go round in circles.''  If the controller is poor, we do not 
need the simulation to be at all accurate in exactly \emph{how} that poor
behaviour is manifest.  We do not need an exact detailed physics model of
the collision between a robot and a wall, or a detailed model of what happens
to a robot arm when it starts oscillating wildly due to a poor control signal.
All we need is for the simulation to be just good enough to indicate that
things have gone wrong, and thus give a low score to that controller.

Given the success of this approach for evolutionary robotics, we propose
that it can be directly used for embodied neuromorphic benchmarks.  To do
this, we create software simulations for each benchmark task.  These simulations
must be fast enough to run in realtime (so that they can be controlled by
real neuromorphic hardware), and they must be extremely variable.  Each time
the simulation is run, different parameters will be chosed for this variability
(so one run might have a large degree of sensor noise while the next run has
none at all; one run might have more delay in the motor response and another
might have less power available).  Being successful at the benchmark means
being successful across all this variability.  

The result should be a benchmark that can be run by any researcher.  The fact
that it is a simulation means that source code can be shared, and that no
specialized hardware is needed.  Furthermore, the variability in the simulation
itself can be controlled, and this can help give a rich characterization of
the benchmarked hardware.  For example, some hardware might only work with
small amounts of sensor noise, or other hardware might only work when there
is significant delay in the motor response.  This flexibility in parameters
in the benchmark allows researchers to explicitly characterize that particular
siuations where their hardware excells.  

\subsection{Cheap Robotics}

All of that said, we also need real physical embodiment as part of any
benchmark of this kind.  While we claim that reliable minimal simulations
are relatively easy to create, they are not a widely-used technique, and the
real proof of behaviour is always in the real world.  For this reason, we
also argue that every benchmark based on a minimal simulation should also
have an easy-to-construct physical analog.  However, this physical version
is not to be used as the primary benchmark.  Rather, this is meant as a
double-check that the system does actually behave as expected.  The worry
is always that if researchers focus on the simulated system when benchmarking
their hardware, they may inadvertantly end up with hardware that only performs
well in simulation.  

For this physical aspect of the benchmark, we recommend cheap, widely-available
components.  This allows a greater chance for other researchers to have
access to the same (or similar) hardware.  For the particular example 
benchmark described in the next section, we use the Lego Mindstorms EV3 kit,
a simple robotics platform available at most toy stores.

It is important to note that there is actually a theoretical advantage to
using cheap robotics hardware for benchmarking, in addition to the practical
advantages.  In particular, we \emph{don't want benchmarks that rely on
particular high-speed, high-accuracy devices}.  The purpose of the benchmark
is not to indicate how well this neuromorphic hardware works to control this
one particular robot in this task.  Rather, the purpose of a benchmark is to
characterise how well this neuromorphic hardware works on this task 
\emph{in general}.  The variability in the minimal simulation means that it
should be able to function across a wide variety of physical embodiments, and
so if we are to choose one particular physical embodiment to test in the
real world, then we should choose one that is not extremely high-precision.
For this reason, we believe using cheap Lego robot is actually more useful
for benchmarking than an expensive high-precision robot.\footnote{Of course,
for more complex benchmark tasks we may need sensory and motor capabiities
that are beyond that of a simple Lego robot.}

\section{A Benchmark: Adaptive Motor Control}

To demonstrate this approach to creating embodied neuromorphic benchmarks,
we now consider a basic control task.  Suppose we have a system with a number
of joints $q$ and we want to send an output $u$ to those motors such that
the joints move to a particular desired position $q_d$.  Our only output is
the signal $u$ (one for each motor) and our only input is the current position
of each motor $q$.

The simplest controller for such a situation is a P (proportional) controller,
where $u=K_p(q_d - q)$.  This is often supplemented with a derivative term,
which helps to slow the system down as it approaches the desired position,
thus avoiding overshooting and oscillation ($u=K_p(q_d - q) + K_d(\dot{q_d} - \dot{q})$),
leading to the standard PD controller.  Both $K_p$ and $K_d$ are constants that can be
hand-tuned to particular situations.

However, this controller has difficulty in the presence of significant external
forces.  For example, consider a single motor controlling the angle of a single
arm.  If the arm is held out to the side, gravity acting on the mass of the
arm itself will pull the arm downward.  Thus to hold the arm still at a
particular $q_d$ will require the controller to apply a force to counteract
gravity.  Since the PD controller always produces an output $u=0$ when $q=q_d$,
it cannot compensate for this. [TODO: add diagram]

The standard solution to this problem is to add an integral term ($K_i \int{q_d-q dt}$)
to the controller, making it a PID controller.  The idea here is that as the difference between where it is and
where we want it to be accumulates over time, the $K-i$ term gradually increases
how much extra force is being applied until it is large enough to counteract
the external force of gravity (or whatever other external forces are present).
However, this approach has great difficulty when $q_d$ changes, since the
external force due to gravity changes depending on the position of the arm $q$.
The controller ends up having to use the accumulated integral to ``relearn'' the
correct amount of extra force needed every time $q_d$ changes.

In some robotics applications, the solution to this problem is to mathematically
analyze the geometry and mass of the system and compute exactly how much extra
force is needed.  In this particular case, the answer is straight-forward,
in that the extra torque due to gravity is $\tau=m l g sin(q) / 2$, where
$m$ is the mass of the arm, $l$ is the length, and $g$ is $9.8m/s^2$.  If the
force applied by the motor is linear in $u$, then we could simply compute this
value and add it to our controller's output.  However, this assumes a perfectly
even distribution of weight in the arm, ignores momentum and other forces, and 
gets much more complex as more joints are added.  Furthermore, if this
initial computation is slightly off, or if details of the system change,
there is no way to adjust this compensation.

Fortunately, there is an adaptive solution to this problem, and it is one that
fits well with neuromorphic hardware.  In \cite{Slotine1987}, Slotine shows that if you
express the influence of these other external forces as $\tau=Y(q) \omega$
(where $Y(q)$ is a fixed set of functions of $q$, such as $sin(q)$, and \omega
is a vector of weights, one for each function in $Y$), then you can learn
to compensate for these external forces by using the learning rule $\Delta \omega = \alpha Y(q) u$,
where $u$ is the basic PD control signal.

Importantly, as pointed out in [ref] and [ref], rather than making explicit
assumptions about the exact functions that should be in $Y(q)$, we can use
a neural network approach where each neuron is a different function of $q$.  
As long as there is enough hetereogenetity (i.e. as long as the neural activity
forms a basis space that spans the desired space of functions), then the
learning rule will continue to work.  

This then suggests an explicit neuromorphic benchmark.  The input to the
neuromorphic hardware is $q$, the system state.  This input is fed to each
neuron such that each neuron produces some output behaviour that is based on
this input.  Since $q$ will be multi-dimensional (if there is more than one
joint), we may give each neuron a random weighting of each $q$ value ($J_i=e_i \cdot q$, 
where $J_i$ is the input to neuron $i$, and $e_i$ is a randomly chosen vector\footnote{$e$ could also be chosen so as to regularly span the space of possibilities}).  Given this input, the neurons will produce some output $A$.
we now form a weighted sum of these outputs $Ad$, where $d$ is a matrix (number of neurons by number of elements in $q$)
that is initially all zeros.

To use this controller, we add its output to that of the standard PD controller.
That is, the standard controller has $u=K_p(q_d - q) + K_d(\dot{q_d} - \dot{q})$,
and so our actual output to the motor is $u + Ad$.  We then apply a learning
rule on $d$ such that $\Delta d = \alpha A \times u$.

Notice that we can think of this system as a three-layer neural
network, where the input and output layers are linear.  The first layer is $q$, the input state, one value for each joint.  The ``hidden'' layer
is $A$, the activity of a large number of neurons.  The output layer again has
one value per joint, and is the extra added signal to apply to the motors, $Ad$.
Given that this is such a canonical example of the use of neural networks, we
hope that the majority of neuromorphic hardware is flexible enough to implement
exactly this model.


\subsection{Online and offline learning}

The one major step here that does not exist in a lot of neuromorphic hardware is
the ability to update the weights $d$.  For hardware that does have a built-in
learning rule, this rule is at least of a very common form, where the weight
update from a neuron is proportional to the activity of that neuron and an
external error signal.  This makes it an instance of the ubiquitous delta rule,
and hopefully supported by the hardware.

However, if the neuromorphic hardware being benchmarked does not have the
ability to update weights online using a learning rule of this form, then there
are two solutions.  First, the multiplication by $d$ could be done on the
output from the neuromorphic hardware.  There has to be some system to take
the neural output from the hardware and send it to the motor (or the simulation
of the motor).  Instead of outputing the result of $Ad$, the hardware could
output $A$ (the activity of all the neurons), and the interface to the motor
could be responsible for doing the multiplication by $d$ and updating $d$
according to the learning rule.

Alternatively, we can use offline learning.  That is, rather than updating
the weights $d$ all the time, we simply record $A$ and $u$, and then after
a period of time stop the controller, compute the sum total of the changes
to $d$, load the new value of $d$ onto the neuromorphic hardware, and then
start the controller again.








% For Original Research Articles, Clinical Trial Articles, and Technology Reports the introduction should be succinct, with no subheadings.
%
% For Clinical Case Studies the Introduction should include symptoms at presentation, physical exams and lab results.
%
Text Text Text Text Text Text  Text Text Text Text Text Text Text Text Text  Text Text Text Text Text Text. Text Text Text Text Text Text  Text Text Text Text Text Text Text Text Text  Text Text Text Text Text Text. Text Text Text Text Text Text  Text Text Text Text Text Text Text Text Text  Text Text Text Text Text Text.Text Text Text Text Text Text  Text Text Text Text Text Text Text Text Text  Text Text Text Text Text Text.

%\begin{methods}
\section{Material \& Methods}

Text Text Text Text Text Text  Text Text Text Text Text Text \cite{conference} Text Text Text  Text Text Text Text Text Text Text Text Text Text  Text Text Text Text Text Text  Text Text.  \cite{patent} might want to know about  text text text text Text Text Text Text  Text Text Text Text Text Text  Text Text. \citep{article} might want to know about  text text text text
Text Text Text Text Text Text  Text Text Text Text Text Text Text Text Text  Text Text Text Text Text Text Text Text Text Text  Text Text Text Text Text Text  Text Text.  \cite{book} might want to know about  text text text text \cite{chapter}

\begin{table}[!t]
\textbf{\refstepcounter{table}\label{Tab:01} Table \arabic{table}.}{ Maximum size of the Manuscript }

\processtable{ }
{\begin{tabular}{lllll}\toprule
 & Abstract max. legth (incl. spaces) & Figures or tables & Manuscript max. length \\\midrule
Clinical Case Study & & & &\\
Clinical Trial & & & &\\
Hypothesis and Theory & & & &\\
Methods & 2000 characters  & 15 & 12000 words \\
Original Research & & & &\\
Review & & & &\\
Technology Report & & & &\\\midrule
Focused Review & 2000 characters & 5 & 5000 words \\\midrule
CPC &  1250 characters& 6 & 2500 words  \\\midrule
Perspective & 1250 characters & 2 & 3000 words  \\
Mini Review & & & &\\\midrule
Data Report & None & 2 & 3000 words\\\midrule
Classification & 1250 characters & 10 & 2000 words \\\midrule
Editorial & None & None & 1000 words  \\\midrule
Frontiers Commentary  & & &\\
General Commentary & None & 1 & 1000 words\\
Book review & & & \\\midrule
Opinion   & & &\\
Specialty Grand Challenge & None & 1 & 2000 words\\
Field Grand Challenge & & & &\\\botrule
\end{tabular}}{}
\end{table}

Please note that large tables covering several pages cannot be included in the final PDF for formatting reasons. These tables will be published as supplementary material on the online article abstract page at the time of acceptance. The author will notified during the typesetting of the final article if this is the case. A link in the final PDF will direct to the online material.

\subsection{Original Research Articles, Clinical Trial Articles, and Technology Reports}

For Original Research Articles, Clinical Trial Articles, and Technology Reports the section headings should be those appropriate for your field and the research itself. It is recommended to organize your manuscript in the following sections or their equivalents for your field:

\begin{itemize}
%for bulleted list, use itemize
\item Introduction: Succinct, with no subheadings.
\item Materials and Methods: This section may be divided by subheadings. This section should contain sufficient detail so that when read in conjunction with cited references, all procedures can be repeated.
\item Results: This section may be divided by subheadings. Footnotes should not be used and have to be transferred into the main text.
\item Discussion: This section may be divided by subheadings. Discussions should cover the key findings of the study: discuss any prior art related to the subject so to place the novelty of the discovery in the appropriate context; discuss the potential short-comings and limitations on their interpretations; discuss their integration into the current understanding of the problem and how this advances the current views; speculate on the future direction of the research and freely postulate theories that could be tested in the future.
\end{itemize}

Please note that the Material and Methods section can be placed in any of the following ways: before Results, before Discussion or after Discussion.

\subsection{Clinical Case Studies}

For Clinical Case Studies the following sections are mandatory:

\begin{itemize}
%for bulleted list, use itemize
\item Introduction: Include symptoms at presentation, physical exams and lab results.
\item Background: This section may be divided by subheadings. Include history and review of similar cases.
\item Results: This section may be divided by subheadings. Include diagnosis and treatment.
\item Concluding Remarks
\end{itemize}

%\end{methods}



\section{Results}

\subsection{Figures}
Frontiers requires figures to be submitted individually, in the same order as they are referred to in the manuscript. Figures will then be automatically embedded at the bottom of the submitted manuscript. Kindly ensure that each table and figure is mentioned in the text and in numerical order. Permission must be obtained for use of copyrighted material from other sources (including the web). Please note that it is compulsory to follow figure instructions. Figures which are not according to the guidelines will cause substantial delay during the production process.


\begin{table}[!t]
\textbf{\refstepcounter{table}\label{Tab:02} Table \arabic{table}.}{ Resolution Requirements for the figures}

\processtable{}
{\begin{tabular}{lllll}\toprule
Image Type & Description & Format & Color Mode & Resolution\\\midrule
Line Art & An image composed of lines and text,  & TIFF, JPEG & RGB, Bitmap & 900 - 1200 dpi\\
           & which does not contain tonal or shaded areas.& & &\\
           Halftone & A continuous tone photograph, which contains no text. & TIFF, EPS, JPEG & RGB, Grayscale & 300 dpi\\
Combination & Image contains halftone + text or line art elements. & TIFF, JPEG & RGB,Grayscale & 600 - 900 dpi\\\botrule
\end{tabular}}{}
\end{table}

\textbf{Table \ref{Tab:02}} shows the resolution requirements for the figures. The figures must be legible:
\begin{enumerate}
\item The smallest visible text is no less than 8 points in height, when viewed at actual size.
\item Solid lines are not broken up.
\item Image areas are not pixelated or stair stepped.
\item Text is legible and of high quality.
\item Any lines in the graphic are no smaller than 2 points width.
\item The actual size of the figure must be of at least 8.5 cm.
\end{enumerate}

\subsection{Nomenclature}
\begin{itemize}
\item The use of abbreviations should be kept to a minimum. Non-standard abbreviations should be avoided unless they appear at least four times, and defined upon first use in the main text. Consider also giving a list of non-standard abbreviations at the end, immediately before the Acknowledgments.
\item Gene symbols should be italicized; protein products are not italicized.
\item Chemical compounds and biomolecules should be referred to using systematic nomenclature, preferably using the recommendations by IUPAC.
\item We encourage the use of Standard International Units in all manuscripts.
\item To take part in the Resource Identification Initiative, please cite antibodies, genetically modified organisms, software tools, data, databases and services using the corresponding catalog number and RRID in your current manuscript. For more information about the project and for steps on how to search for an RRID, please click \href{http://www.frontiersin.org/files/pdf/letter_to_author.pdf}{here}.
\end{itemize}

\begin{equation}
\sum x+ y =Z\label{eq:01}
\end{equation}

\section{Discussion}

Text Text Text Text Text Text  Text Text Text Text Text Text Text Text Text  Text Text Text Text Text Text Text Text Text Text.
Additional Requirements:
\subsection{Corrections}

If you need to communicate important changes to a published article please submit a General Commentary. Submit the article with the title “Corrigendum: Original Title of Article”.

\subsection{Commentaries on Articles}

At the beginning of your Commentary, please provide the citation of the article commented on. Rebuttals may be submitted in response to Commentaries; our limit in place is one commentary and one response. Rebuttals should also be submitted as General Commentary articles.

\subsection{Human Search and Animal Research}

All experiments on live vertebrates or higher invertebrates must be performed in accordance with relevant institutional and national guidelines and regulations. In the manuscript, authors must identify the committee approving the experiments and must confirm that all experiments conform to the relevant regulatory standards. For manuscripts reporting experiments on human subjects, authors must identify the committee approving the experiments and must also include a statement confirming that informed consent was obtained from all subjects. In Original Research Articles and Clinical Trial Articles these statements should appear in the Materials and Methods section.

\subsection{Clinical Trial Registration}

Clinical trials should be registered in a public trials registry in order to become the object of a publication at Frontiers. Trials must be registered at or before the start of patient enrollment. A clinical trial is defined as "any research study that prospectively assigns human participants or groups of humans to one or more health-related interventions to evaluate the effects on health outcomes."(\href{www.who.int/ictrp/en}{www.who.int/ictrp/en}). A list of acceptable registries can be found at \href{www.who.int/ictrp/en}{www.who.int/ictrp/en} and \href{www.icmje.org}{www.icmje.org}.

\subsection{Inclusion of Proteomics Data}

Authors should provide relevant information relating to how the peptide/protein matches were undertaken, including methods used to process and analyze data, false discovery rates (FDR) for large-scale studies and threshold or cut-off rates for peptide and protein matches. Further information could include software used, mass spectrometer type, sequence database and version, number of sequences in database, processing methods, mass tolerances used for matching, variable/fixed modifications, allowable missed cleavages, etc.

Authors should provide as supplementary material information used to identify proteins and/or peptides. This should include information such as accession numbers, observed mass (m/z), charge, delta mass, matched mass, peptide/protein scores, peptide modification, miscleavages, peptide sequence, match rank, matched species (for cross species matching), number of peptide matches, ambiguous protein/peptide matches should be indicated, etc.
For quantitative proteomics analyses authors should provide information to justify the statistical significance including biological replicates, statistical methods, estimates of uncertainty and the methods used for calculating error.

For peptide matches with biologically relevant post-translational modifications (PTM) and for any protein match that has occurred using a single mass spectrum, authors should include this information as raw data, annotated spectra or submit data to an online repository (recommended option).
Authors are encouraged to submit raw or matched data and 2-DE images to public proteomics repositories. Submission codes and/or links to data should be provided within the manuscript.

\subsection{Data Sharing}

Frontiers supports the policy of data sharing, and authors are advised to make freely available any materials and information described in their article, and any data relevant to the article (while not compromising confidentiality in the context of human-subject research) that may be reasonably requested by others for the purpose of academic and non-commercial research. In regards to deposition of data and data sharing through databases, Frontiers urges authors to comply with the current best practices within their discipline.

\section*{Disclosure/Conflict-of-Interest Statement}
%Frontiers follows the recommendations by the International Committee of Medical Journal Editors (http://www.icmje.org/ethical_4conflicts.html) which require that all financial, commercial or other relationships that might be perceived by the academic community as representing a potential conflict of interest must be disclosed. If no such relationship exists, authors will be asked to declare that the research was conducted in the absence of any commercial or financial relationships that could be construed as a potential conflict of interest. When disclosing the potential conflict of interest, the authors need to address the following points:
%•	Did you or your institution at any time receive payment or services from a third party for any aspect of the submitted work?
%•	Please declare financial relationships with entities that could be perceived to influence, or that give the appearance of potentially influencing, what you wrote in the submitted work.
%•	Please declare patents and copyrights, whether pending, issued, licensed and/or receiving royalties relevant to the work.
%•	Please state other relationships or activities that readers could perceive to have influenced, or that give the appearance of potentially influencing, what you wrote in the submitted work.

The authors declare that the research was conducted in the absence of any commercial or financial relationships that could be construed as a potential conflict of interest.

\section*{Author Contributions}
%When determining authorship the following criteria should be observed:
%•	Substantial contributions to the conception or design of the work; or the acquisition, analysis, or interpretation of data for the work; AND
%•	Drafting the work or revising it critically for important intellectual content; AND
%•	Final approval of the version to be published ; AND
%•	Agreement to be accountable for all aspects of the work in ensuring that questions related to the accuracy or integrity of any part of the work are appropriately investigated and resolved.
%Contributors who meet fewer than all 4 of the above criteria for authorship should not be listed as authors, but they should be acknowledged. (http://www.icmje.org/roles_a.html)

The statement about the authors and contributors can be up to several sentences long, describing the tasks of individual authors referred to by their initials and should be included at the end of the manuscript before the References section.


\section*{Acknowledgments}
Text Text Text Text Text Text  Text Text Text Text Text Text Text Text  Text Text Text Text Text Text Text Text Text  Text Text Text. Text Text Text Text Text Text  Text Text Text Text Text Text Text Text  Text Text Text Text Text Text Text Text Text  Text Text Text. 


\textit{Funding\textcolon} Text Text Text Text Text Text  Text Text.

\section*{Supplemental Data}
Supplementary Material should be uploaded separately on submission, if there are Supplementary Figures, please include the caption in the same file as the figure. LaTeX Supplementary Material templates can be found in the Frontiers LaTeX folder

Text Text Text Text Text Text  Text Text Text Text Text Text Text Text  Text Text Text Text Text Text Text Text Text  Text Text Text.


\bibliographystyle{frontiersinSCNS_ENG_HUMS} % for Science, Engineering and Humanities and Social Sciences articles, for Humanities and Social Sciences articles please include page numbers in the in-text citations
%\bibliographystyle{frontiersinHLTH&FPHY} % for Health and Physics articles
\bibliography{paper}

%%% Upload the *bib file along with the *tex file and PDF on submission if the bibliography is not in the main *tex file

\section*{Figures}

%%% Use this if adding the figures directly in the mansucript, if so, please remember to also upload the files when submitting your article
%%% There is no need for adding the file termination, as long as you indicate where the file is saved. In the examples below the files (logo1.jpg and logo2.eps) are in the Frontiers LaTeX folder
%%% If using *.tif files convert them to .jpg or .png

\begin{figure}[h!]
\begin{center}
\includegraphics[width=10cm]{logo1}% This is a *.jpg file
\end{center}
 \textbf{\refstepcounter{figure}\label{fig:01} Figure \arabic{figure}.}{ Enter the caption for your figure here.  Repeat as  necessary for each of your figures }
\end{figure}

%\begin{figure}
%\begin{center}
%\includegraphics[width=10cm]{logo2}% This is an *.eps file
%\end{center}
%\textbf{\refstepcounter{figure}\label{fig:02} Figure \arabic{figure}.}{ Enter the caption for your figure here.  Repeat as  necessary for each of your figures }
%\end{figure}

%%% If you don't add the figures in the LaTeX files, please upload them when submitting the article.

%%% Frontiers will add the figures at the end of the provisional pdf automatically %%%

%%% The use of LaTeX coding to draw Diagrams/Figures/Structures should be avoided. They should be external callouts including graphics.

\end{document}
